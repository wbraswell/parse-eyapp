The Parrot virtual machine \cite{parrot} is register based. This means that, like a hardware
CPU, it has a number of fast-access units of storage called registers.
There are 4 types of register in Parrot: integers (\verb|I|), numbers (\verb|N|),
strings (\verb|S|) and PMCs (\verb|P|). For each type there are several of these, named
\verb|$N0|, \verb|$N1|, \ldots  Number registers map
to the machine native floating point type. 
You can download a recent version of parrot from
\htmladdnormallink{http://www.parrot.org/download}{http://www.parrot.org/download}.

The code produced by \verb|infix2pir.pl| is an example of PIR, which stands for 
Parrot Intermediate Representation and is also known as
Intermediate Code or IMC. PIR files use the extension \verb|.pir|.
PIR is an intermediate language that can be compiled to
Parrot Byte code (PBC). It was conceived as a possible target language for compilers
targeting the Parrot Virtual Machine. PIR is halfway between a High
Level Language (HLL) and Parrot Assembly (PASM).

PIR has a relatively simple syntax. Every line is a comment, a
label, a statement, or a directive. Each statement or directive
stands on its own line. There is no end-of-line symbol (such as a
semicolon in C). These is a brief enumeration of the main characteristics of the language:

\begin{itemize}
\item {\bf Comments }
A comment begins with the \verb|#| symbol, and continues until the end of
the line. Comments can stand alone on a line or follow a statement
or directive.

\item {\bf Statements }

A statement is either an opcode or syntactic sugar for one or more
opcodes. An opcode is a native instruction for the virtual machine;
it consists of the name of the instruction followed by zero or more
arguments.

\item {\bf Higher-level constructs }

PIR also provides higher-level constructs, including symbolic operators:

\begin{verbatim}
  $I1 = 2 + 5
\end{verbatim}

These special statement forms are just syntactic sugar for regular
opcodes. The \verb|+| symbol corresponds to the \verb|add| opcode, the 
\verb|-| symbol
to the \verb|sub| opcode, and so on. The previous example is equivalent
to:

\begin{verbatim}
  add $I1, 2, 5
\end{verbatim}

\item {\bf Directives }

Directives resemble opcodes, but they begin with a period (\verb|.|). Some
directives specify actions that occur at compile time. Other
directives represent complex operations that require the generation
of multiple instructions. The \verb|.local| directive, for example, declares
a named variable.

\begin{verbatim}
  .local string hello
\end{verbatim}

\item {\bf Literals }

\begin{itemize}
\item {\bf Numbers }

Integers and floating point numbers are numeric literals. They can be positive or negative.
Integer literals can also be binary, octal, or hexadecimal:

\begin{verbatim}
  $I0 = 42       # positive
  $I1 = -1       # negative
  $I1 = 0b01010  # binary
  $I2 = 0o72     # octal
  $I3 = 0xA5     # hexadecimal
\end{verbatim}

Floating point number literals have a decimal point and can use scientific notation:

\begin{verbatim}
  $N0 = 3.14
  $N2 = -1.2e+4
\end{verbatim}

\item {\bf Strings }

String literals are enclosed in single or double-quotes.
Strings in double-quotes allow escape sequences using backslashes.
Strings in single-quotes only allow escapes for nested quotes

\begin{verbatim}
  $S0 = "This is a valid literal string"
  $S1 = 'This is also a valid literal string'
\end{verbatim}
\end{itemize}

\item {\bf Variables }

PIR variables can store four different kinds of values—integers,
numbers (floating point), strings, and objects. Parrot's objects
are called PMCs, for "PolyMorphic Container".

The simplest kind of variable is a register variable. The name of
a register variable always starts with a dollar sign (\verb|$|), followed
by a single character which specifies the type of the variable --
integer (\verb|I|), number (\verb|N|), string (\verb|S|), or PMC (\verb|P|) 
-- and ends with a unique number. You need not predeclare register variables:

\begin{verbatim}
  $S0 = "Who's a pretty boy, then?"
  say $S0
\end{verbatim}

PIR also has named variables; the \verb|.local| directive declares them.
As with register variables, there are four valid types: \verb|int|, \verb|num|,
\verb|string|, and \verb|pmc|. You must declare named variables; otherwise they
behave exactly the same as register variables.

\begin{verbatim}
  .local string hello
  hello = "'Allo, 'allo, 'allo."
  say hello
\end{verbatim}

\item {\bf Constants }

The \verb|.const| directive declares a named constant. Named constants are
similar to named variables, but the values set in the declaration
may never change. Like \verb|.local|, \verb|.const| takes a type and a name. It
also requires a literal argument to set the value of the constant.

\begin{verbatim}
  .const int    frog = 4                       # integer
  .const string name = "Superintendent Parrot" # string
  .const num    pi   = 3.14159                 # floating point
\end{verbatim}

You may use a named constant anywhere you may use a literal, but
you must declare the named constant beforehand. 

\item {\bf Keys }

A key is a special kind of constant used for accessing elements in
complex variables (such as an array). A key is either an integer
or a string; and it's always enclosed in square brackets 
(\verb|[| and \verb|]|).
You do not have to declare literal keys. This code example stores
the string \verb|"foo"|
in \verb|$P0| as element \verb|5|, and then retrieves it.

\begin{verbatim}
  $P0[5] = "foo"
  $S1    = $P0[5]
\end{verbatim}

PIR supports multi-part keys. Use a semicolon to separate each part.

\begin{verbatim}
  $P0['my';'key'] = 472
  $I1             = $P0['my';'key']
\end{verbatim}

\item {\bf Control Structures }

\begin{itemize}
\item {\bf goto}

The \verb|goto label| statement 
jumps to a named label. It can only
jump inside a subroutine, and only to a named label. 
Example:

\begin{verbatim}
    goto GREET
      # ... some skipped code ...
  GREET:
    say "'Allo, 'allo, 'allo."
\end{verbatim}

\item {\bf if }

Variations on the basic \verb|goto| check whether a particular condition
is true or false before jumping:

\begin{verbatim}
  if $I0 > 5 goto GREET
\end{verbatim}
\end{itemize}

\item {\bf Subroutines }

A PIR subroutine starts with the \verb|.sub| directive and ends with the
\verb|.end| directive. 
Parameter declarations use the \verb|.param| directive;
they resemble named variable declarations. 

This example declares a
subroutine named greeting, that takes a single string parameter
named \verb|hello|:

\begin{verbatim}
  .sub 'greeting'
      .param string hello
      say hello
  .end
\end{verbatim}
\end{itemize}

